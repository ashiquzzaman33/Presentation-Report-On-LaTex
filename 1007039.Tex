\documentclass[a4paper,12pt]{article}
\pagenumbering{arabic}
\usepackage{graphicx}
\DeclareGraphicsExtensions{.pdf,.png,.jpg,.eps}
\usepackage{amsrefs}

\begin{document}

    \begin{center}

        \begin{figure}[htp]
        \begin{center}
        \includegraphics[scale=0.25]{jpg/The-Cloud.jpg}
        \end{center}
        %\caption{} \label{}
        \end{figure}

        \begin{tabular}{c}
            \hline
            \hline
            \\ \\
 	{\LARGE\textbf{A Social Compute Cloud: Allocating}}\\
            {\LARGE\textbf{and Sharing Infrastructure Resources}}\\
	 {\LARGE\textbf{ via Social Networks}}\\
            \\
            \textbf{Simon Caton}, Member, IEEE,  \textbf{Christian Haas}, Member, IEEE\\
	  \textbf{ Kyle Chard}, Member, IEEE,  \textbf{Kris Bubendorfer}, Member, IEEE\\
 	   and  \textbf{ Omer F. Rana}, Member, IEEE\\
            \\ \\
            \hline
	\hline
            \\ \\
        \end{tabular}

    \end{center}

     {\large\textbf{Published In :}}\\

    \begin{center}
        \begin{tabular}{c}
            \textbf{Magazine :}\\
            Services Computing, IEEE Transactions on\\
            (Volume:7, Issue:3)\\
            \\
            \textbf{Date of Publication :}\\
            July-Sept. 2014 \\
            \\
            \textbf{ISSN :}\\
            1939-1374\\
            \\
            \textbf{DOI :}\\
            10.1109/TSC.2014.2303091
        \end{tabular}

    \end{center}

    \newpage
    {\large\textbf{Submitted By :}}

    \begin{center}
        \begin{tabular}{c}
            \\
            \textbf{Course No :}\\
            CSE - 4120\\
            \\
            \textbf{Course Name :}\\
            Technical Writing \& Seminar\\
            \\
            \textbf{Name :}\\
            MD. ASHIQUZZAMAN\\
            \\
            \textbf{Roll :}\\
            1107039\\
            \\
            \textbf{Date of Submission :}\\
            Jun  16 , 2015\\
            \\ \\
            Department of Computer Science\\
                    \&\\
            Engineering, Khulna - 9203.\\
            \\
        \end{tabular}

        \begin{figure}[htp]
        \begin{center}
        \includegraphics[scale=.42]{png/kuet.png}
        \end{center}
        %\caption{} \label{}
        \end{figure}

    \end{center}

    \newpage
    \tableofcontents
    \listoffigures
    \newpage
    %%%Abstract%%%
    %\addcontentsline{toc}{section}{Abstract}
    \section{Abstract}
	Social network platforms have rapidly changed the way that people communicate and interact. They have enabled the
	establishment of, and participation in, digital communities as well as the representation, documentation and exploration of social
	relationships. We believe that as ‘apps’ become more sophisticated, it will become easier for users to share their own services,
	resources and data via social networks. To substantiate this, we present a social compute cloud where the provisioning of cloud
	infrastructure occurs through “friend” relationships. In a social compute cloud, resource owners offer virtualized containers on their
	personal computer(s) or smart device(s) to their social network. However, as users may have complex preference structures
	concerning with whom they do or do not wish to share their resources, we investigate, via simulation, how resources can be effectively
	allocated within a social community offering resources on a best effort basis. In the assessment of social resource allocation, we
	consider welfare, allocation fairness, and algorithmic runtime. The key findings of this work illustrate how social networks can be
	leveraged in the construction of cloud computing infrastructures and how resources can be allocated in the presence of user sharing
	preferences.
   %%%Index Terms%%%
    \section{Index Terms}
	\textbf{Social cloud computing, social networks, cloud computing, preference-based resource allocation}
    %%%Introduction%%%
    \section{Introduction}
            Identity verification, that is, confirming that an identity is real and that the individual claiming the identity is entitled to it, presents something of a privacy conundrum.
            If an identity claim isn't properly verified, entities that rely on it might become victims of fraud or improperly disclose personal information, violating the individual�s privacy.
            So, to access personal information for a transaction, must ensure identity verification.

            \begin{figure}[htp]
                \begin{center}
                    \includegraphics[height=1.5in,width=2.5in]{jpg/main.jpg}
                \end{center}
                \caption{Identity Verification.} \label{Fig:1}
            \end{figure}

    \newpage
%%%Objectives%%%
    \section{A SOCIA LCOMPUTECLOUD}
        The objective of this article is to enhance the identity services, verify identification and ensure privacy to users.
        Followings are the objectives of this article :

        \begin{itemize}

            \item {\subsection*{Verification of Identity :}}
                Verify the identity of a requesting person and the authority of any such person to have access to it.
                If the identity or any such authority of such person is not known to the covered entity, then disclosures this.
                \textbf{Such that} - Personal Intellectual Representation.
                Let, identity documents that can be verified include :
                {\begin{itemize}
                    \item {ID - Identification Card}
                    \item {SSN - Social Security Number}
                    \item {Etc.}
                \end{itemize}}

            \item {\subsection*{Evidences of Identity :}}
                Request should accept depending on any relevant evidence that appears reliable and reasonable under the circumstances.
                \textbf{Such that} - Authorization To Access (permission or birthday).
                Let, identity documents that can be verified include :
                {\begin{itemize}
                    \item {Driver Licences}
                    \item {Passports}
                    \item {Etc.}
                \end{itemize}}

            \item {\subsection*{Optional Identity :}}
                Verification is not required for appropriate actions if there is reasonable belief of safety another individual.
                \textbf{Such that} - Public Official Area.
                Let, identity documents that can be verified include :
                {\begin{itemize}
                    \item {User Id \& Password}
                    \item {Security Question \& Answer}
                    \item {Etc.}
                \end{itemize}}

        \end{itemize}

    \newpage
	

\end{document} 